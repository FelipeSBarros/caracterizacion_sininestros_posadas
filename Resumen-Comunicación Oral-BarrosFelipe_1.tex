\documentclass[11pt]{article}

\oddsidemargin=-0cm  
\evensidemargin=0cm 
\topmargin=-1cm
\textwidth=16cm
\textheight=24cm
\headsep=1.5cm

%----------------PAQUETES----------------------------

\usepackage[spanish]{babel}
\usepackage[utf8]{inputenc}
\usepackage[centertags]{amsmath}
\usepackage{amsfonts}
\usepackage{graphicx}
\usepackage{mathrsfs}
%\pagestyle{headings}
\usepackage{amssymb}
\usepackage{amsthm}
\usepackage{booktabs}
\usepackage{xcolor}
\usepackage[hidelinks]{hyperref}
\usepackage{fancyhdr}


%ENCABEZADO
\chead{LII COLOQUIO ARGENTINO DE ESTADÍSTICA 2025}
\renewcommand{\headrulewidth}{0pt}

%%%%%%%%%%%%%%%%%%%%%%%%%%%%%%%%%%%
\pagestyle{fancy}


\begin{document}

%%%% Título, Autores y filiaciones
\begin{center}
\begin{Large}
\textbf{TÍTULO DEL TRABAJO}
\end{Large}
\vspace{0.6cm}

Felipe Sodré Mendes Barros$^1$$^2$$^3$ 

\vspace{0.6cm}
\textit{$^1$\  Facultad de Ciencias Forestales (FCF)-Universidad Nacional de Misiones}\\
\textit{$^2$\ Área de Herramientas de Soporte a las Decisiones (LabHSD)}\\
\textit{$^3$\ Laboratorio de Herramientas de Soporte a las Decisiones (LabHSD)}

\vspace{0.6cm}
\textit{correo@electrónico de cada autor o expositor en el nombre de presentación anterior separados por comas}
\end{center}

\begin{center}
\vspace{0.9cm}
\begin{Large}
\textbf{RESUMEN}
\end{Large}
\end{center}
%%%%%%%%%%%%%%%%%%%%%%%%%%
%% Aquí comienza su trabajo %%%
El resumen debe describir brevemente el trabajo indicando objetivos, metodología y conclusiones, no excediendo las 300 palabras con formato justificado. Se puede usar esta plantilla para crear el resumen reemplazando sus textos. \\
Los NOMBRES DE LOS AUTORES deben escribirse en mayúsculas y usar superíndices para hacer referencia a las afiliaciones institucionales de los mismos – no hay límite para la cantidad de autores. Dichas afiliaciones se escriben en letra minúscula cursiva, así como también la dirección de correo electrónico del expositor. 
%%%% Hasta aquí %%%%

\vspace{0.9cm}

\noindent \textbf{Palabras Clave:} \textit{de tres a seis palabras con formato itálica minúscula.}

\vspace{0.9cm}

\noindent \textbf{Área Temática: \textit{Estadística esapacial}}

\end{document}

