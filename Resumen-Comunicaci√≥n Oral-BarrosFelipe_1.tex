\documentclass[11pt]{article}

\oddsidemargin=-0cm  
\evensidemargin=0cm 
\topmargin=-1cm
\textwidth=16cm
\textheight=24cm
\headsep=1.5cm

%----------------PAQUETES----------------------------

\usepackage[spanish]{babel}
\usepackage[utf8]{inputenc}
\usepackage[centertags]{amsmath}
\usepackage{amsfonts}
\usepackage{graphicx}
\usepackage{mathrsfs}
%\pagestyle{headings}
\usepackage{amssymb}
\usepackage{amsthm}
\usepackage{booktabs}
\usepackage{xcolor}
\usepackage[hidelinks]{hyperref}
\usepackage{fancyhdr}


%ENCABEZADO
\chead{LII COLOQUIO ARGENTINO DE ESTADÍSTICA 2025}
\renewcommand{\headrulewidth}{0pt}

%%%%%%%%%%%%%%%%%%%%%%%%%%%%%%%%%%%
\pagestyle{fancy}


\begin{document}

%%%% Título, Autores y filiaciones
\begin{center}
\begin{Large}
\textbf{TÍTULO DEL TRABAJO}
\end{Large}
\vspace{0.6cm}

FELIPE SODRÉ MENDES BARROS$^1$$^2$$^3$, JONATHAN VON BELOW$^1$  

\vspace{0.6cm}
\textit{$^1$\  Facultad de Ciencias Forestales (FCF)-Universidad Nacional de Misiones}\\
\textit{$^2$\ Área de Herramientas de Soporte a las Decisiones}\\
\textit{$^3$\ Laboratorio de Herramientas de Soporte a las Decisiones (LabHSD)}

\vspace{0.6cm}
\textit{felipe.sodre@fcf.unam.edu.ar, jonathan.vonbelow@fcf.unam.edu.ar }
\end{center}

\begin{center}
\vspace{0.9cm}
\begin{Large}
\textbf{RESUMEN}
\end{Large}
\end{center}
%%%%%%%%%%%%%%%%%%%%%%%%%%
%% Aquí comienza su trabajo %%%
La siniestralidad vial representa un grave problema de salud pública en Argentina, con tasas elevadas en provincias del noreste como Misiones. Ante la carencia de datos oficiales abiertos en ciudades intermedias como Posadas, este estudio propone una alternativa metodológica basada en fuentes periodísticas y análisis espacial de precisión. A partir de noticias publicadas por el diario Primera Edición entre 2022 y 2023, se elaboró una base de datos georreferenciada de siniestros viales, complementada con información sobre semáforos provenientes de la Infraestructura de Datos Espaciales de Posadas.\\
Se aplicaron técnicas de estadística espacial para analizar los siniestros como procesos puntuales. El análisis de primer orden mediante estimación de densidad kernel reveló una distribución espacial inhomogénea, con alta concentración en avenidas como Quaranta, Uruguay, Mitre y Junín. La estimación de segundo orden, función L(d), permitió identificar correlación espacial significativa a distancias de 100 a 500 metros, evidenciando patrones de aglomeración en corredores viales críticos. Estos hallazgos guiaron la aplicación del algoritmo DBSCAN, que identificó hotspots en intersecciones clave de la ciudad, particularmente en zonas con alta circulación y presencia de motociclistas. Además, se examinó la relación espacial entre siniestros y semáforos mediante análisis bivariado, observándose una inhibición significativa de los siniestros a más de 350 metros de estos dispositivos, lo que sugiere un efecto protector en su entorno cercano.\\
Los resultados evidencian el potencial de combinar fuentes alternativas con herramientas geoespaciales para generar evidencia empírica en contextos donde los datos oficiales son limitados. Este enfoque metodológico puede contribuir a la formulación de políticas públicas basadas en evidencia y adaptadas a las particularidades viales de ciudades intermedias latinoamericanas. 
%%%% Hasta aquí %%%%

\vspace{0.9cm}

\noindent \textbf{Palabras Clave:} \textit{ESTADÍSTICA ESPACIAL, GEORREFERENCIACIÓN, SINIESTROS VIALES, DENSIDAD DE KERNEL, FUNCIÓN L DE RIPLEY, DBSCAN}

\vspace{0.9cm}

\noindent \textbf{Área Temática: \textit{Estadística espacial}}

\end{document}

